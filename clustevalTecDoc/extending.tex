
	\section{Extending the Framework}\label{sec_extend}
	\clusteval can be extended in different ways. The following sections will show you, which functionality you can add to the framework and how.
	\subsection{Clustering Methods}\label{subsec_extend_programs}
	As explained in \ref{programs} \clusteval supports two different kinds of clustering methods: Standalone programs and R programs. Standalone programs are those, for which you have to provide an executable file which then will be executed by the framework. R programs are methods implemented in R, which will be invoked by \clusteval by using the Rserve interface.
	\subsubsection{Standalone Programs}\label{subsubsec_extend_standprograms}
	Standalone programs can be added to \clusteval by
	\begin{enumerate}
		\item putting the executable file (together with all required shared libraries) into a respective folder in the repository programs directory
		
\highlight{\repoprogramexec}
		\item putting a program configuration file (see \ref{subsec_programconfigs}) into the repository program configuration directory
		
\highlight{\repoprogramconfigs}
		\item if the program requires a new input format, follow the instructions under \ref{subsec_extend_datasetformats} for the new input format
		\item if the program has an unknown output format, follow the instructions under \ref{subsec_extend_runresultformats} for the new output format
	\end{enumerate}
	\subsubsection{R Programs}\label{subsubsec_extend_rprograms}
	R programs can be added to \clusteval by
	\begin{enumerate}
		\item extending the class \highlight{de.clusteval.program.r.RProgram} with your own class \highlight{MyRProgram}. You have to provide your own implementations for the following methods, otherwise the framework will not be able to load your class.
		\begin{enumerate}
			\item \highlight{public MyRProgram(Repository)}: The constructor of your class taking a repository parameter. This constructor has to be implemented and public, otherwise the framework will not be able to load your class.
			\item \highlight{public MyRProgram(MyRProgram)}: The copy constructor of your class taking another instance of your class. This constructor has to be implemented and public.
			\item \highlight{public String getAlias()}: This alias is used whenever this program is visually represented and a readable name is needed. This is used to represent your program on the website for example.
			\item \highlight{public String getInvocationFormat()}: This is the invocation of the R method including potential parameters, that have to be defined in the program configuration.
			\item \highlight{public Set getRequiredLibraries()}: This method returns the set of strings, with names of all required R libraries this program uses.
			\item \highlight{public Process exec(DataConfig,ProgramConfig,String[],Map,Map)}: In this method the actual execution of the R Program happens. In here you have to implement the invocation of the R method via Rserve and any pre- and postcalculations necessary. The communications with R can be visualized by the following code snippet:
			\begin{verbatim}
try {
    // precalculations
    double[] input = ...;
    ...
    MyRengine rEngine = new MyRengine("");
    try {
        rEngine.assign("input",input);
        rEngine.eval("result <- yourMethodInvocation()");
        REXP result = rEngine.eval("result@.Data");
        
        // postcalculations
        ...
    } catch (RserveException e) {
        e.printStackTrace();
    } finally {
        rEngine.close();
    }
} catch (Exception e) {
    e.printStackTrace();
}
// for return type compatibility reasons 
return null;
				
			\end{verbatim}		
		\end{enumerate}
			\item Creating a jar file named \highlight{MyRProgram.jar} containing the \highlight{MyRProgram.class} compiled on your machine in the correct folder structure corresponding to the packages:
			
			\code{de/clusteval/program/r/MyRProgram.class}
			\item Putting the \highlight{MyRProgram.jar} into the programs folder of the repository:
			
			\highlight{\repoprograms}
			
			The backend server will recognize and try to load the new program automatically the next time, the \highlight{RProgramFinderThread} checks the filesystem.
			
	\end{enumerate}
	\subsection{Dataset Types}\label{subsec_extend_datasettypes}
	Dataset types can be added to \clusteval by
	\begin{enumerate}
		\item  extending the class \highlight{de.clusteval.data.dataset.type.DataSetType} with your own class \highlight{MyDataSetType}. You have to provide your own implementations for the following methods, otherwise the framework will not be able to load your class.
		\begin{enumerate}
			\item \highlight{public MyDataSetType(Repository, boolean,long, File)}: The constructor of your class. This constructor has to be implemented and public, otherwise the framework will not be able to load your class.
			\item \highlight{public MyDataSetType(MyDataSetType)}: The copy constructor of your class taking another instance of your class. This constructor has to be implemented and public.
			\item \highlight{public String getAlias()}: This alias is used whenever this program is visually represented and a readable name is needed. This is used to represent your program on the website for example.
		\end{enumerate}
			\item Creating a jar file named \highlight{MyDataSetType.jar} containing the \highlight{MyDataSetType.class} compiled on your machine in the correct folder structure corresponding to the packages:
			
			\code{de/clusteval/data/dataset/type/MyDataSetType.class}
			\item Putting the \highlight{MyDataSetType.jar} into the dataset types folder of the repository:
			
			\highlight{\reposuppdstypes}
			
			The backend server will recognize and try to load the new dataset type automatically the next time, the \highlight{DataSetTypeFinderThread} checks the filesystem.
	\end{enumerate}
	\subsection{Dataset Formats}\label{subsec_extend_datasetformats}
	A dataset format \highlight{MyDataSetFormat} can be added to \clusteval by
	\begin{enumerate}
		\item extending the class \highlight{de.clusteval.data.dataset.format.DataSetFormat} with your own class \highlight{MyDataSetFormat}. You have to provide your own implementations for the following methods, otherwise the framework will not be able to load your dataset format.
		\begin{enumerate}
			\item \highlight{public MyDataSetFormat(Repository, boolean, long, File, int)}: The constructor of your dataset format class. This constructor has to be implemented and public, otherwise the framework will not be able to load your dataset format.
			\item \highlight{public MyDataSetFormat(MyDataSetFormat)}: The copy constructor of your class taking another instance of your class. This constructor has to be implemented and public.
		\end{enumerate}
		\item extending the class \highlight{de.clusteval.data.dataset.format.DataSetFormatParser} with your own class \highlight{MyDataSetFormatParser}. You have to provide your own implementations for the following methods, otherwise the framework will not be able to load your class.
		\begin{enumerate}
			\item \highlight{public DataSet convertToStandardFormat(DataSet,
			
			\hspace{10mm}ConversionInputToStandardConfiguration)}: This method converts the given dataset to the standard input format of the framework using the given conversion configuration. This assumes, that the passed dataset has this format.
			\item \highlight{public DataSet convertToThisFormat(DataSet, DataSetFormat, 
			
			\hspace{10mm}ConversionConfiguration)}: This method converts the given dataset to the given input format using the conversion configuration.
			\item \highlight{public Object parse(DataSet)}:  This method parses the given dataset and returns an object, wrapping the contents of the dataset (e.g. an instance of \highlight{SimilarityMatrix} or \highlight{DataMatrix}).
		\end{enumerate}
			\item Creating a jar file named \highlight{MyDataSetFormat.jar} containing the \highlight{MyDataSetFormat.class} and \highlight{MyDataSetFormatParser.class} compiled on your machine in the correct folder structure corresponding to the packages:
			
			\code{de/clusteval/data/dataset/format/MyDataSetFormat.class}
			
			\code{de/clusteval/data/dataset/format/MyDataSetFormatParser.class}
			\item Putting the \highlight{MyDataSetFormat.jar} into the dataset formats folder of the repository:
			
			\highlight{\reposuppinputformats}
			
			The backend server will recognize and try to load the new dataset format automatically the next time, the \highlight{DataSetFormatFinderThread} checks the filesystem.
	\end{enumerate}
	\subsection{Runresult Formats}\label{subsec_extend_runresultformats}
	A runresult format \highlight{MyRunResultFormat} can be added to \clusteval by
	\begin{enumerate}
		\item extending the class \highlight{de.clusteval.run.result.format.RunResultFormat} with your own class \highlight{MyRunResultFormat}. You have to provide your own implementations for the following methods, otherwise the framework will not be able to load your runresult format.
		\begin{enumerate}
			\item \highlight{public MyRunResultFormat(Repository, boolean, long, File)}: The constructor of your runresult format class. This constructor has to be implemented and public, otherwise the framework will not be able to load your runresult format.
			\item \highlight{public MyRunResultFormat(MyRunResultFormat)}: The copy constructor of your class taking another instance of your class. This constructor has to be implemented and public.
		\end{enumerate}
		\item extending the class \highlight{de.clusteval.run.result.format.RunResultFormatParser} with your own class \highlight{MyRunResultFormatParser}. You have to provide your own implementations for the following methods, otherwise the framework will not be able to load your class.
		\begin{enumerate}
			\item {\footnotesize \highlight{public void convertToStandardFormat()}}: This method converts the given runresult to the standard runresult format of the framework. The converted runresult has to be named exactly as the input file postfixed with the extension ".conv". The original runresult 
			
			{\footnotesize \highlight{\reporesult/clusters/TransClust\_sfld.1.result}}
			
			has to be converted to
			
			{\footnotesize \highlight{\reporesult/clusters/TransClust\_sfld.1.result.conv}}
			
			by this method. A wrapper object for the converted runresult has to be stored in the result attribute.
		\end{enumerate}
			\item Creating a jar file named \highlight{MyRunResultFormat.jar} containing the \highlight{MyRunResultFormat.class} and \highlight{MyRunResultFormatParser.class} compiled on your machine in the correct folder structure corresponding to the packages:
			
			\code{de/clusteval/run/result/format/MyRunResultFormat.class}
			
			\code{de/clusteval/run/result/format/MyRunResultFormatParser.class}
			
			\item Putting the \highlight{MyRunResultFormat.jar} into the runresult formats folder of the repository:
			
			\highlight{\reposuppresultformats}
			
			The backend server will recognize and try to load the new runresult format automatically the next time, the \highlight{RunResultFormatFinderThread} checks the filesystem.
	\end{enumerate}
	
	
	\subsection{Parameter Optimization Methods}\label{subsec_extend_paramoptmethods}
	A parameter optimization method \highlight{MyParameterOptimizationMethod} can be added to \clusteval by
	\begin{enumerate}
		\item extending the class \highlight{de.clusteval.cluster.paramOptimization.ParameterOptimizationMethod} with your own class \highlight{MyParameterOptimizationMethod}. You have to provide your own implementations for the following methods, otherwise the framework will not be able to load your parameter optimization method.
		\begin{enumerate}
			\item \highlight{public MyParameterOptimizationMethod(Repository, boolean, long, File, 
			
			\hspace{10 mm}ParameterOptimizationRun, ProgramConfig, DataConfig, List,
			
			\hspace{10 mm}ClusteringQualityMeasure, int[], boolean)}: The constructor for your parameter optimization method. This constructor has to be implemented and public, otherwise the framework will not be able to load your parameter optimization method.
			\item \highlight{public MyParameterOptimizationMethod( MyParameterOptimizationMethod)}: The copy constructor for your parameter optimization method. This constructor has to be implemented and public, otherwise the framework will not be able to load your parameter optimization method.
			\item \highlight{public List getCompatibleDataSetFormatBaseClasses()}: A list of dataset formats, this parameter optimization method can be used for. If the list is empty, all dataset formats are assumed to be compatible.
			\item \highlight{public List getCompatibleProgramNames()}: A list of names of all programs that are compatible to this parameter optimization method. If the list is empty, all programs are assumed to be compatible.
			\item \highlight{public boolean hasNext()}: This method indicates, whether their is another parameter set to evaluate. This method \textbf{must not change} the current parameter set, as it may be invoked several times before next() is invoked.
			\item \highlight{protected ParameterSet getNextParameterSet(ParameterSet)}: Returns the next parameter set to evaluate. This method may change the internal status of the parameter optimization method, in that it stores the newly determined and returned parameter set as the current parameter set.
			\item \highlight{public Set getRequiredRlibraries()}: Returns a set of names of all R libraries, this parameter optimization method requires.
			\item \highlight{public int getTotalIterationCount()}: This is the total iteration count this parameter optimization method will perform. The returned value might not correspond to the expected value, when the method is instantiated. Therefore always use the return value of this method, when trying to determine the finished percentage of the parameter optimization process.
		\end{enumerate}
		
			\item Creating a jar file named \highlight{MyParameterOptimizationMethod.jar} containing the \highlight{MyParameterOptimizationMethod.class} compiled on your machine in the correct folder structure corresponding to the packages:
			
			\code{de/clusteval/cluster/paramOptimization/MyParameterOptimizationMethod.class}
			
			\item Putting the \highlight{MyParameterOptimizationMethod.jar} into the parameter optimization methods folder of the repository:
			
			\highlight{\reposuppparamopts}
			
			The backend server will recognize and try to load the new parameter optimization method automatically the next time, the \highlight{ParameterOptimizationMethodFinderThread} checks the filesystem.
		\end{enumerate}
			
			
	\subsection{Distance Measures}\label{subsec_extend_distmeasures}
	A distance measure \highlight{MyDistanceMeasure} can be added to \clusteval by
	\begin{enumerate}
		\item extending the class \highlight{de.clusteval.data.distance.DistanceMeasure} with your own class \highlight{MyDistanceMeasure}. You have to provide your own implementations for the following methods, otherwise the framework will not be able to load your distance measure.
		\begin{enumerate}
			\item \highlight{public MyDistanceMeasure(Repository, boolean, long, File)}: The constructor for your distance measure. This constructor has to be implemented and public, otherwise the framework will not be able to load your distance measure.
			\item \highlight{public MyDistanceMeasure(MyDistanceMeasure)}: The copy constructor for your distance measure. This constructor has to be implemented and public, otherwise the framework will not be able to load your distance measure.
			\item \highlight{public double getDistance(double[],double[])}: This method is the core of your distance measure. It returns the distance of the two points specified by the absolute coordinates in the two double arrays.
			\item \highlight{public boolean supportsMatrix()}: This method indicates, whether your distance measure can calculate distances of a whole set of point-pairs, i.e. your distance measure implements the method getDistances(double[][]).
			\item \highlight{public double[][] getDistances(double[][])}:  The absolute coordinates of the points are stored row-wise in the given matrix and distances are calculated between every pair of rows. Position [i][j] of the returned double[][] matrix contains the distance between the i-th and j-th row of the input matrix.
			\item \highlight{public Set getRequiredRlibraries()}: Returns a set of names of all R libraries, this distance measure requires.
		\end{enumerate}
		
			\item Creating a jar file named \highlight{MyDistanceMeasure.jar} containing the \highlight{MyDistanceMeasure.class} compiled on your machine in the correct folder structure corresponding to the packages:
			
			\code{de/clusteval/data/distance/MyDistanceMeasure.class}
			
			\item Putting the \highlight{MyDistanceMeasure.jar} into the distance measure folder of the repository:
			
			\highlight{\reposuppdistmeasures}
			
			The backend server will recognize and try to load the new distance measure automatically the next time, the \highlight{DistanceMeasureFinderThread} checks the filesystem.
		\end{enumerate}
		
	\subsection{Clustering Quality Measures}\label{subsec_extend_qualitymeasures}
	A clustering quality measure \highlight{MyClusteringQualityMeasure} can be added to \clusteval by
	\begin{enumerate}
		\item extending the class \highlight{de.clusteval.cluster.quality.ClusteringQualityMeasure} with your own class \highlight{MyClusteringQualityMeasure}. You have to provide your own implementations for the following methods, otherwise the framework will not be able to load your clustering quality measure.
		\begin{enumerate}
			\item \highlight{public MyClusteringQualityMeasure(Repository, boolean, long, File)}: The constructor for your distance measure. This constructor has to be implemented and public, otherwise the framework will not be able to load your clustering quality measure.
			\item \highlight{public MyClusteringQualityMeasure(MyClusteringQualityMeasure)}: The copy constructor for your distance measure. This constructor has to be implemented and public, otherwise the framework will not be able to load your clustering quality measure.
			\item \highlight{public String getAlias()}: This method returns a readable alias for this clustering quality measure which is used e.g. on the website.
			\item \highlight{public double getMinimum()}: Returns the minimal value this measure can calculate.
			\item \highlight{public double getMaximum()}: Returns the maximal value this measure can calculate.
			\item \highlight{public boolean requiresGoldStandard()}: Indicates, whether this clustering quality measure requires a goldstandard to assess the quality of a given clustering.
			\item \highlight{public ClusteringQualityMeasureValue getQualityOfClustering(Clustering)}: This method is the core of your clustering quality measure. It assesses and returns the quality of the given clustering.
			\item \highlight{public boolean isBetterThanHelper(ClusteringQualityMeasureValue}: This method is used by sorting algorithms of the framework to compare clustering quality measure results and find the optimal parameter sets.
			\item \highlight{public Set getRequiredRlibraries()}: Returns a set of names of all R libraries, this clustering quality measure requires.
		\end{enumerate}
		
			\item Creating a jar file named \highlight{MyClusteringQualityMeasure.jar} containing the \highlight{MyClusteringQualityMeasure.class} compiled on your machine in the correct folder structure corresponding to the packages:
			
			\code{de/clusteval/cluster/quality/MyClusteringQualityMeasure.class}
			
			\item Putting the \highlight{MyClusteringQualityMeasure.jar} into the clustering quality measure folder of the repository:
			
			\highlight{\reposuppqualmeasures}
			
			The backend server will recognize and try to load the new clustering quality measure automatically the next time, the \highlight{ClusteringQualityMeasureFinderThread} checks the filesystem.
		\end{enumerate}
	\subsection{Data Statistics}\label{subsec_extend_datastats}
	A data statistic \highlight{MyDataStatistic} can be added to \clusteval by
	\begin{enumerate}
		\item extending the class \highlight{de.clusteval.data.statistic.DataStatistic} with your own class \highlight{MyDataStatistic}. You have to provide your own implementations for the following methods, otherwise the framework will not be able to load your class.
		\begin{enumerate}
			\item \highlight{public MyDataStatistic(Repository, boolean, long, File)}: The constructor for your data statistic. This constructor has to be implemented and public, otherwise the framework will not be able to load your class.
			\item \highlight{public MyDataStatistic(MyDataStatistic)}: The copy constructor for your data statistic. This constructor has to be implemented and public, otherwise the framework will not be able to load your class.
			\item \highlight{public boolean requiresGoldStandard()}: Indicates, whether this data statistic requires a goldstandard to assess the property of a given dataset.
			\item \highlight{public String getAlias()}: This method returns a readable alias for this data statistic which is used e.g. on the website.
			\item \highlight{public void parseFromString(String)}: This method interprets the string and fills this statistic object with its parsed contents.
			\item \highlight{public Set getRequiredRlibraries()}: Returns a set of names of all R libraries, this data statistic requires.
		\end{enumerate}
		\item extending the class \highlight{de.clusteval.data.statistic.DataStatisticCalculator} with your own class \highlight{MyDataStatisticCalculator}. You have to provide your own implementations for the following methods, otherwise the framework will not be able to load your class.
		\begin{enumerate}
			\item \highlight{public MyDataStatisticCalculator(Repository, long, File, DataConfig)}: The constructor for your data statistic calculator. This constructor has to be implemented and public, otherwise the framework will not be able to load your class.
			\item \highlight{public MyDataStatisticCalculator(MyDataStatisticCalculator)}: The copy constructor for your data statistic calculator. This constructor has to be implemented and public, otherwise the framework will not be able to load your class.
			\item \highlight{protected MyDataStatistic calculateResult()}: This method is the core of your data statistic calculator. It analysis the given data configuration and returns a wrapper object for the results.
			\item \highlight{public void writeOutputTo(File)}: After calculateResult() has been invoked, this method writes the assessed results into the given file.
		\end{enumerate}
		
			\item Creating a jar file named \highlight{MyDataStatisticCalculator.jar} containing the \highlight{MyDataStatistic.class} and \highlight{MyDataStatisticCalculator.class} compiled on your machine in the correct folder structure corresponding to the packages:
			
			\code{de/clusteval/data/statistics/MyDataStatistic.class}
			
			\code{de/clusteval/data/statistics/MyDataStatisticCalculator.class}
			
			\item Putting the \highlight{MyDataStatistic.jar} into the data statistics folder of the repository:
			
			\highlight{\reposuppdatastats}
			
			The backend server will recognize and try to load the new data statistics automatically the next time, the \highlight{DataStatisticFinderThread} checks the filesystem.
		\end{enumerate}
		
		
	\subsection{Run Statistics}\label{subsec_extend_runstats}
	A run statistic \highlight{MyRunStatistic} can be added to \clusteval by
	\begin{enumerate}
		\item extending the class \highlight{de.clusteval.run.statistic.RunStatistic} with your own class \highlight{MyRunStatistic}. You have to provide your own implementations for the following methods, otherwise the framework will not be able to load your class.
		\begin{enumerate}
			\item \highlight{public MyRunStatistic(Repository, boolean, long, File)}: The constructor for your run statistic. This constructor has to be implemented and public, otherwise the framework will not be able to load your class.
			\item \highlight{public MyRunStatistic(MyRunStatistic)}: The copy constructor for your run statistic. This constructor has to be implemented and public, otherwise the framework will not be able to load your class.
			\item \highlight{public String getAlias()}: This method returns a readable alias for this run statistic which is used e.g. on the website.
			\item \highlight{public void parseFromString(String)}: This method interprets the string and fills this statistic object with its parsed contents.
			\item \highlight{public Set getRequiredRlibraries()}: Returns a set of names of all R libraries, this run statistic requires.
		\end{enumerate}
		\item extending the class \highlight{de.clusteval.run.statistic.RunStatisticCalculator} with your own class \highlight{MyRunStatisticCalculator}. You have to provide your own implementations for the following methods, otherwise the framework will not be able to load your class.
		\begin{enumerate}
			\item \highlight{public MyRunStatisticCalculator(Repository, long, File, DataConfig)}: The constructor for your run statistic calculator. This constructor has to be implemented and public, otherwise the framework will not be able to load your class.
			\item \highlight{public MyRunStatisticCalculator(MyRunStatisticCalculator)}: The copy constructor for your run statistic calculator. This constructor has to be implemented and public, otherwise the framework will not be able to load your class.
			\item \highlight{protected MyRunStatistic calculateResult()}: This method is the core of your run statistic calculator. It analysis the given runresults and returns a wrapper object for the results.
			\item \highlight{public void writeOutputTo(File)}: After calculateResult() has been invoked, this method writes the assessed results into the given file.
		\end{enumerate}
		
			\item Creating a jar file named \highlight{MyRunStatisticCalculator.jar} containing the \highlight{MyRunStatistic.class} and \highlight{MyRunStatisticCalculator.class} compiled on your machine in the correct folder structure corresponding to the packages:
			
			\code{de/clusteval/run/statistics/MyRunStatistic.class}
			
			\code{de/clusteval/run/statistics/MyRunStatisticCalculator.class}
			
			\item Putting the \highlight{MyRunStatistic.jar} into the run statistics folder of the repository:
			
			\highlight{\reposupprunstats}
			
			The backend server will recognize and try to load the new run statistics automatically the next time, the \highlight{RunStatisticFinderThread} checks the filesystem.
		\end{enumerate}
		
		
	\subsection{Run-Data Statistics}\label{subsec_extend_rundatastats}
	A run-data statistic \highlight{MyRunDataStatistic} can be added to \clusteval by
	\begin{enumerate}
		\item extending the class \highlight{de.clusteval.run.statistic.RunDataStatistic} with your own class \highlight{MyRunDataStatistic}. You have to provide your own implementations for the following methods, otherwise the framework will not be able to load your class.
		\begin{enumerate}
			\item \highlight{public MyRunDataStatistic(Repository, boolean, long, File)}: The constructor for your run-data statistic. This constructor has to be implemented and public, otherwise the framework will not be able to load your class.
			\item \highlight{public MyRunDataStatistic(MyRunDataStatistic)}: The copy constructor for your run-data statistic. This constructor has to be implemented and public, otherwise the framework will not be able to load your class.
			\item \highlight{public String getAlias()}: This method returns a readable alias for this run-data statistic which is used e.g. on the website.
			\item \highlight{public void parseFromString(String)}: This method interprets the string and fills this statistic object with its parsed contents.
			\item \highlight{public Set getRequiredRlibraries()}: Returns a set of names of all R libraries, this run-data statistic requires.
		\end{enumerate}
		\item extending the class \highlight{de.clusteval.run.statistic.RunDataStatisticCalculator} with your own class \highlight{MyRunDataStatisticCalculator}. You have to provide your own implementations for the following methods, otherwise the framework will not be able to load your class.
		\begin{enumerate}
			\item \highlight{public MyRunDataStatisticCalculator(Repository, long, File, DataConfig)}: The constructor for your run-data statistic calculator. This constructor has to be implemented and public, otherwise the framework will not be able to load your class.
			\item \highlight{public MyRunDataStatisticCalculator(MyRunDataStatisticCalculator)}: The copy constructor for your run-data statistic calculator. This constructor has to be implemented and public, otherwise the framework will not be able to load your class.
			\item \highlight{protected MyRunDataStatistic calculateResult()}: This method is the core of your run-data statistic calculator. It analysis the given runresults (data analysis and clustering) and returns a wrapper object for the results.
			\item \highlight{public void writeOutputTo(File)}: After calculateResult() has been invoked, this method writes the assessed results into the given file.
		\end{enumerate}
		
			\item Creating a jar file named \highlight{MyRunDataStatisticCalculator.jar} containing the \highlight{MyRunDataStatistic.class} and \highlight{MyRunDataStatisticCalculator.class} compiled on your machine in the correct folder structure corresponding to the packages:
			
			\code{de/clusteval/run/statistics/MyRunDataStatistic.class}
			
			\code{de/clusteval/run/statistics/MyRunDataStatisticCalculator.class}
			
			\item Putting the \highlight{MyRunDataStatistic.jar} into the run-data statistics folder of the repository:
			
			\highlight{\reposupprundatastats}
			
			The backend server will recognize and try to load the new run-data statistics automatically the next time, the \highlight{RunDataStatisticFinderThread} checks the filesystem.
		\end{enumerate}