	\section{Introduction}\label{sec_intro}
	In the next chapter we will discuss briefly how \clusteval is organized, how it works in principle and what can be achieved through its use.
	
	Chapter \ref{sec_inst} explains, how the framework is to be installed and, amongst other things, what prerequisites are necessary to install and use it successfully. For Debian 6 (Squeeze) we provide a detailed step-by-step installation manual. For those of you, who do not want to install it manually but want to try out the framework right away, we offer a VirtualBox hard drive image which contains a Debian 6 installation, together with the framework and all its prerequisites (see \ref{subsec_vbox_image}).
	
	Chapter \ref{sec_usage} demonstrates the usage of the framework by a simple use case scenario from the very start to the very end. If you want more detailed information about the use cases \clusteval assists with, please  consult \cite{wiwie_2013}.
	
	If you are a programmer and you want to extend \clusteval, you will find more information in section \ref{sec_extend}.