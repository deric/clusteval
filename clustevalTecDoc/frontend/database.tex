
	\section{Frontend Database}\label{subsec_frontend_db}
	The MySQL database of the frontend stores a subset of the data contained in the repository of the backend. The stored information can then be retrieved and visualized by the website.
	 
	\subsection{Tables}
	In the following we will give a short description of every table of the mysql database. Table names in square brackets correspond to rather technical tables, that are just required by the models of the ruby on rails website and do not have a strong meaning with regard to contents.

\begin{figure}[hbtp]
\caption{Inheritance of different run types (database hierarchy)}
\label{fig_run_inheritance_tables}
\centering
\includegraphics[width=.6\textwidth]{../master_seminar_presentation/run_inheritance_tables.png}
\end{figure} 
	
	\begin{description}
	\item[Hint 1]: All tables that correspond to and are responsible for storing repository objects have a foreign key to the repository table. Thus for every repository object it is known, to which repository it belongs. This is not mentioned in the following descriptions.
	
	\item[Hint 2]: The abbreviation FK means Foreign Key.
	
	\item[Hint 3]: Column names are denoted in italic.
	
	\item[Hint 4]: When a run is performed, certain files are copied into a new result folder. This includes datasets, goldstandards and all configuration files. These files in the result folder are mapped to their original files they correspond to in the original repository. These relationships are stored in the database in a separate column which is named after the table name plus the postfix "\_id". For example datasets store this relationship in the column "dataset\_id".
	\end{description}

	\subsubsection{[aboutus]}
	A technical table containing the information regarding the 'About us' section of the website.
	\subsubsection{[admins]}
	A technical table containing the information regarding the 'Admin' section of the website.
	\subsubsection{cluster\_objects}
	Every clustering contains clusters, which again contain cluster objects. This table stores all cluster objects with their \textit{name} and knows, to which cluster they belong (\textit{cluster\_id}, FK).

		\subsubsection{clustering\_quality\_measure\_optimums} For visualization and interpretation of results the website needs to know, whether a certain clustering quality measure is optimal when min- or maximized. This table maps every measure to 'Minimum' or 'Maximum' (\textit{name}).
		
		\subsubsection{clustering\_quality\_measures                                  } This table keeps track of all the clustering quality measures available in the framework. For every measure it stores its \textit{name}, minimal and maximal value (\textit{minValue} and \textit{maxValue}) and whether the measure requires a goldstandard (\textit{requiresGoldStandard}). On the website every clustering quality measure has a readable alias. This alias is stored in the column (\textit{alias}).
	
		
		\subsubsection{clusterings                                                  } This table holds all the clusterings that were calculated and are stored in the repository. Every parsed clustering corresponds to a file in the repository, for which we store its absolute path (\textit{absPath}).
		
		\subsubsection{clusters                                                   } Every clustering has clusters. This table holds all clusters and maps them to their corresponding clustering. Every cluster has a \textit{name}.
		
		\subsubsection{data\_configs                                                 } A table holding all data configurations. Every data configuration has an absolute path (\textit{absPath}), a \textit{name}, a corresponding dataset configuration (\textit{dataset\_config\_id}, FK) and a goldstandard configuration (\textit{goldstandard\_config\_id}, FK).
		
		\subsubsection{dataset\_configs                                              } A table holding all dataset configurations. Every dataset configuration has an absolute path (\textit{absPath}), a \textit{name} and a corresponding dataset (\textit{dataset\_id}, FK).
		
		\subsubsection{dataset\_formats                                              } A table holding all dataset formats. Every format has a \textit{name} and an \textit{alias}.
		
		\subsubsection{dataset\_types                                                } This table holds all types of datasets. For every dataset type it stores a \textit{name}. On the website every dataset type has a readable alias. For every dataset type this table stores the \textit{alias}.
		
		\subsubsection{datasets                                                     } This table holds all datasets. For every dataset its absolute path (\textit{absPath}), \textit{checksum}, format (\textit{dataset\_format\_id}, FK) and type (\textit{dataset\_type\_id}, FK) is stored. 
		
		\subsubsection{goldstandard\_configs                                         } This table holds all goldstandard configurations. For every goldstandard configuration its absolute path (\textit{absPath}), \textit{name} and corresponding goldstandard (\textit{goldstandard\_id}, FK) is stored.
		
		\subsubsection{goldstandards}
		This table holds all goldstandards. For every goldstandard its absolute path (\textit{absPath}), and the corresponding goldstandard (\textit{goldstandard\_id}, FK) is stored.
		
		\subsubsection{[helps]}
	A technical table containing the information regarding the 'Help' section of the website.
	
		\subsubsection{[mains]}
	A technical table containing the information regarding the startpage of the website.
	
		\subsubsection{optimizable\_program\_parameters                               }
	This table stores the program parameters (\textit{program\_parameter\_id}, FK) of a program configuration (\textit{program\_config\_id}, FK), that can be optimized.
	
		\subsubsection{parameter\_optimization\_methods}
	This table stores all available parameter optimization methods (\textit{name}) registered in a repository (\textit{repository\_id}, FK).
	
	\subsubsection{program\_configs                                              } All program configurations registered in a repository are stored in this table. Every program configuration has a \textit{name}, an absolute path (\textit{absPath}), different invocation formats for different scenarios (\textit{invocationFormat}, \textit{invocationFormatWithoutGoldStandard}, \textit{invocationFormatParameterOptimization}, \textit{invocationFormatParameterOptimizationWithoutGoldStandard}) and a boolean whether the program expects input with normalized similiarites (\textit{expectsNormalizedDataSet}). For every program configuration we store the corresponding repository (\textit{repository\_id}, FK), the program (\textit{program\_id}, FK) this configuration belongs to, the run result format (\textit{run\_result\_format\_id}, FK) of the program using this configuration. 
	
	\subsubsection{program\_configs\_compatible\_dataset\_formats                   } Every program configuration (\textit{program\_config\_id}, FK) has a set of compatible dataset formats (\textit{dataset\_format\_id}, FK), which the program will understand when it is executed using this configuration.
	
	\subsubsection{[program\_descriptions]}
	This table stores descriptions of clustering methods, for when they are shown on the website.
	
	\subsubsection{[program\_images]}
	When this table contains an image for a clustering method, it will be shown on the website.
	
	\subsubsection{program\_parameter\_types}
	This table contains the names (\textit{name}) of the different possible program parameter types (see \ref{subsec_programconfigs} for more information on the types of parameters supported by \clusteval).
	
	\subsubsection{program\_parameters}
	This table stores the program parameters defined in a program configuration (\textit{program\_config\_id}, FK). Every program parameter has a type (\textit{program\_parameter\_type\_id}, FK), a \textit{name}, an (optional) \textit{description}, a \textit{minValue}, a \textit{maxValue} and a default value (\textit{def}).
		
	\subsubsection{[program\_publications]}		
	When this table contains publication information for a clustering method, it will be shown on the website.
	
	\subsubsection{programs}
	This tables stores all clustering methods together with their \textit{absPath} and an \textit{alias}, which is used to represent this clustering method on the website.
	
	\subsubsection{repositories}
	The repositories that are using this database to store their results. Every repository has a absolute base directory (\textit{basePath}) and a type (\textit{repository\_type\_id}, FK).
		
	\subsubsection{repository\_types}
	The types that repositories can have. Every type has a \textit{name}. Check out \ref{subsec_repository} for more information on which repository types exist.
	
	\subsubsection{run\_analyses}
	This table holds all analysis runs. Every analysis run is also a run (\textit{run\_id}, FK).

\begin{figure}[hbtp]
\caption{Foreign keys of analysis runs}
\label{fig_analysis_runs_foreign_keys}
\centering
\includegraphics[scale=.4]{../master_seminar_presentation/db_analysis_runs.png}
\end{figure} 
	
	\subsubsection{run\_analysis\_statistics}
	Every analysis run (\textit{run\_analysis\_id}, FK) evaluates certain statistics (\textit{statistic\_id}, FK).
	
	\subsubsection{run\_clusterings}
	This table holds all clustering runs. Every clustering run is also a execution run (\textit{run\_execution\_id}, FK).
	
	\subsubsection{run\_data\_analyses}
	This table holds all data analysis runs. Every data analysis run is also an analysis run (\textit{run\_analysis\_id}, FK).

\begin{figure}[hbtp]
\caption{Foreign keys of data analysis runs}
\label{fig_data_analysis_runs_foreign_keys}
\centering
\includegraphics[scale=.4]{../master_seminar_presentation/db_data_analysis_runs.png}
\end{figure} 
	
	\subsubsection{run\_data\_analysis\_data\_configs}
	Every data analysis run analysis a set of data configurations wrapping datasets. This table holds the data configurations (\textit{data\_config\_id}, FK) that a certain data analysis run (run\_data\_analysis\_id, FK) analyses.
	
	\subsubsection{run\_execution\_data\_configs}
	An execution run applies program configurations to data configurations. This table stores the data configurations (\textit{data\_config\_id}, FK) belonging to execution runs (\textit{run\_execution\_id}, FK).
	
	\subsubsection{run\_execution\_parameter\_values}
	An execution run can specify values for program parameters. This table stores for every execution run (\textit{run\_execution\_id}, FK), program configuration (\textit{program\_config\_id}, FK) and program parameter (\textit{program\_parameter\_id}, FK) the specified \textit{value}.
	
	\subsubsection{run\_execution\_program\_configs}
	An execution run applies program configurations to data configurations. This table stores the program configurations (\textit{program\_config\_id}, FK) belonging to execution runs (\textit{run\_execution\_id}, FK).
	
	\subsubsection{run\_execution\_quality\_measures}
	An execution run applies clustering methods to datasets and assesses clustering quality measures. This table stores the execution run (run\_execution\_id, FK) together with the clustering quality measures (\textit{clustering\_quality\_measure\_id}, FK) to assess.
	
	\subsubsection{run\_executions}
	This table holds all execution runs. Every execution run is also a run (\textit{run\_id}, FK).

\begin{figure}[hbtp]
\caption{Foreign keys of execution runs}
\label{fig_execution_runs_foreign_keys}
\centering
\includegraphics[scale=.4]{../master_seminar_presentation/db_execution_runs.png}
\end{figure} 
	
	\subsubsection{run\_internal\_parameter\_optimizations}
	This table holds all internal parameter optimization runs. Every such run is also an execution run (\textit{run\_execution\_id}, FK).
	
	\subsubsection{run\_parameter\_optimization\_methods}
	A parameter optimization run uses parameter optimization methods to optimize parameters. For a certain parameter optimization run (\textit{run\_parameter\_optimization\_id}, FK) for every program configuration (\textit{program\_config\_id}, FK) a different parameter optimization method (\textit{parameter\_optimization\_method\_id}, FK) and a clustering quality measure to optimize can be specified.
	
	\subsubsection{run\_parameter\_optimization\_parameters}
	A parameter optimization run optimizes parameters of clustering methods. This table stores for a certain run (\textit{run\_parameter\_optimization\_id}, FK) for every program configuration contained (\textit{program\_config\_id}, FK) the parameters (\textit{program\_parameter\_id}, FK) to optimize.
	
	\subsubsection{run\_parameter\_optimization\_quality\_measures}
	A parameter optimization run optimizes parameters by maximizing or minimizing clustering quality measures. This table stores all clustering quality measures (\textit{clustering\_quality\_measure\_id}, FK) to assess for the calculated clusterings.
	
	\subsubsection{run\_parameter\_optimizations}
	This table holds all parameter optimization runs. Every parameter optimization run is also an execution run (\textit{run\_execution\_id}, FK).

\begin{figure}[hbtp]
\caption{Foreign keys of parameter optimization runs}
\label{fig_parameter_optimization_runs_foreign_keys}
\centering
\includegraphics[scale=.4]{../master_seminar_presentation/db_parameter_optimization_runs.png}
\end{figure} 
	
	\subsubsection{run\_result\_data\_analysis\_data\_configs\_statistics}
	A data analysis run assesses statistics for certain data configurations. This table stores the assessed statistics (\textit{statistic\_id}, FK) for every run result (\textit{run\_result\_id}, FK) generated by a 
	
	\subsubsection{run\_result\_formats}
	This table holds all run result formats. Every run result format has a \textit{name}.
	
	\subsubsection{run\_results}
	When a run is executed, it produces a unique folder in the results directory of the repository. These folders are stored in this table together with the type of the corresponding run (\textit{run\_id}, FK) that created this result (\textit{run\_type\_id}, FK), an absolute path to the results folder (\textit{absPath}), the \textit{uniqueRunIdentifier} of this run result (which corresponds to the name of the folder) and the \textit{date} the run result was created.
	
	\subsubsection{run\_results\_analyses}
	When an analysis run is executed, it produces a unique folder in the results directory of the repository. Every analysis run result is also a run result (\textit{run\_result\_id}, FK).
	
	\subsubsection{run\_results\_clusterings}
	When a clustering run is executed, it produces a unique folder in the results directory of the repository. Every clustering run result is also an execution run result (\textit{run\_results\_execution\_id}, FK).
	
	\subsubsection{run\_results\_data\_analyses}
	When a data analysis run is executed, it produces a unique folder in the results directory of the repository. Every data analysis run result is also an analysis run result (\textit{run\_results\_analysis\_id}, FK).
	
	\subsubsection{run\_results\_executions}
	When an execution run is executed, it produces a unique folder in the results directory of the repository. Every execution run result is also a run result (\textit{run\_result\_id}, FK).
	
	\subsubsection{run\_results\_internal\_parameter\_optimizations}
	When an internal parameter optimization run is executed, it produces a unique folder in the results directory of the repository. Every internal parameter optimization run result is also an execution run result (\textit{run\_results\_execution\_id}, FK).
	
	\subsubsection{run\_results\_parameter\_optimizations}
	When a parameter optimization run is executed, it produces a set of iterative run results, that are all summarized in .complete-files for every pair of program and data configuration (\textit{program\_config\_id}, FK) and (\textit{data\_config\_id}, FK). Every parameter optimization run result is also an execution run result (\textit{run\_results\_execution\_id}, FK).
	
	\subsubsection{run\_results\_parameter\_optimizations\_parameter\_set\_iterations}
	Every parameter optimization produces clustering results for a set of iterations. In each iteration a different parameter set (\textit{run\_results\_parameter\_optimizations\_parameter\_set\_id}, FK) is evaluated. This table holds the number of the \textit{iteration}, together with the parameter set, the produced clustering (\textit{clustering\_id}, FK) and the parameter set in a string representation (\textit{paramSetAsString}).
	
	\subsubsection{run\_results\_parameter\_optimizations\_parameter\_set\_parameters}
	This table holds the program parameters (\textit{program\_parameter\_id}, FK) that belong to parameter sets (\textit{run\_results\_parameter\_optimizations\_parameter\_set\_id}, FK) contained in a parameter optimization run result (\textit{run\_results\_parameter\_optimization\_id}, FK), evaluated by the framework.
	
	\subsubsection{run\_results\_parameter\_optimizations\_parameter\_sets}
	This table holds the parameter sets evaluated during a parameter optimization run and contained in a parameter optimization run result (\textit{run\_results\_parameter\_optimization\_id}, FK).
	
	\subsubsection{run\_results\_parameter\_optimizations\_parameter\_values}
	This table contains the \textit{value} of a certain program parameter (\textit{run\_results\_parameter\_optimizations\_parameter\_set\_parameter\_id}, FK) evaluated in a certain parameter optimization iteration (\textit{run\_results\_parameter\_optimizations\_parameter\_set\_iteration\_id}, FK).
	
	\subsubsection{run\_results\_parameter\_optimizations\_qualities}
	In every iteration of a parameter optimization a parameter set is evaluated and clustering qualities are assessed. This table holds the clustering quality measure (\textit{cluster\_quality\_measure\_id}, FK) together with the assessed \textit{quality}.
	
	\subsubsection{run\_results\_run\_analyses}
	When a run analysis run is executed, it produces a unique folder in the results directory of the repository. Every run analysis run result is also an analysis run result (\textit{run\_results\_analysis\_id}, FK).
	
	\subsubsection{run\_results\_run\_data\_analyses}
	When a run-data analysis run is executed, it produces a unique folder in the results directory of the repository. Every run-data analysis run result is also an analysis run result (\textit{run\_results\_analysis\_id}, FK).
	
	\subsubsection{run\_run\_analyses}
	This table holds all run analysis runs. Every run analysis run is also an analysis run (\textit{run\_analysis\_id}, FK).

\begin{figure}[hbtp]
\caption{Foreign keys of run analysis runs}
\label{fig_run_analysis_runs_foreign_keys}
\centering
\includegraphics[scale=.4]{../master_seminar_presentation/db_run_analysis_runs.png}
\end{figure} 
	
	\subsubsection{run\_run\_analysis\_run\_identifiers}
	Every run analysis run analyses (\textit{run\_run\_analysis\_id}, FK) a set of run results with certain identifiers (\textit{runIdentifier}).
	
	\subsubsection{run\_run\_data\_analyses}
	This table holds all run-data analysis runs. Every run-data analysis run is also an analysis run (\textit{run\_analysis\_id}, FK).

\begin{figure}[hbtp]
\caption{Foreign keys of run-data analysis runs}
\label{fig_run_data_analysis_runs_foreign_keys}
\centering
\includegraphics[scale=.4]{../master_seminar_presentation/db_run_data_analysis_runs.png}
\end{figure} 
	
	\subsubsection{run\_run\_data\_analysis\_data\_identifiers}
	Every run-data analysis run analyzes (\textit{run\_run\_analysis\_id}, FK) a set of data analysis run results with certain identifiers (\textit{dataIdentifier}).
	
	\subsubsection{run\_run\_data\_analysis\_run\_identifiers}
	Every run-data analysis run analyzes (\textit{run\_run\_analysis\_id}, FK) a set of execution run results with certain identifiers (\textit{runIdentifier}).
	
	\subsubsection{run\_types}
	This table holds all types of runs. Every type has a \textit{name}. Check out \ref{fig_run_inheritance} for more information on which run types exist.
	
	\subsubsection{runs}
	This table holds all runs. Every run has a type (\textit{run\_type\_id}, FK), an absolute path (\textit{absPath}), a \textit{name} and a \textit{status}.
	
	\subsubsection{[schema\_migrations]}
	This table holds all the migrations of the ruby on rails website.
	
	\subsubsection{statistics}
	This table holds all the statistics registered in a repository. Every statistic has a \textit{name} and an \textit{alias}. The alias is used on the website as a readable name.
	
	\subsubsection{statistics\_datas}
	This table holds all the data statistics. Every data statistic is also a statistic (\textit{statistic\_id}, FK).
	\subsubsection{statistics\_runs}
	This table holds all the run statistics. Every run statistic is also a statistic (\textit{statistic\_id}, FK).
	\subsubsection{statistics\_run\_datas}
	This table holds all the run-data statistics. Every run-data statistic is also a statistic (\textit{statistic\_id}, FK).
	
	\subsubsection{[submit\_datasets]}
	This table corresponds to the section "Submit Dataset" of the website.
	\subsubsection{[submit\_methods]}
	This table corresponds to the section "Submit Clustering Method" of the website.
	\subsubsection{[submits]}
	This table corresponds to the "Submit" section of the website.
	\subsubsection{[users]}
	This table holds all the users, that have registered on the website.